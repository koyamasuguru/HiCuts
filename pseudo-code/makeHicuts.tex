
%定義

\documentclass[10pt]{jarticle} 
%\usepackage{graphicx}
%\usepackage{fancybox}
%\usepackage{comment}
\usepackage{amsmath}
\usepackage{amssymb}
\usepackage{amsfonts}
\usepackage[commentsnumbered,longend,linesnumbered]{algorithm2e}
\newcommand{\forcond}{$i=0$ \KwTo $n$}
\SetKwFunction{FRecurs}{FnRecursive}%
\SetKwProg{Fn}{Function}{}{end}\SetKwFunction{FRecurs}{makeHicuts}



\pagestyle{empty}

%余白とか

\setlength{\topmargin}{-3.0cm} 
\setlength{\textheight}{28.0cm} 
\setlength{\textwidth}{18.5cm}
\setlength{\oddsidemargin}{-1.3cm} 
\setlength{\columnsep}{.5cm}
\newcommand{\noin}{\noindent}
\catcode`@=\active \def@{\hspace{0.9bp}-\hspace{0.9bp}}
\newtheorem{dfn}{定義}[section]


%タイトル

\title{HiCuts決定木構成の擬似コード}
\setcounter{footnote}{1}
\author{小山 卓\if0\thanks{神奈川大学理学部情報科学科 田中研究室}\fi}
\date{\today}
\西暦

%タイトル作成

\begin{document}

\maketitle
\thispagestyle{empty}

\begin{algorithm}
\Fn(\tcc*[h]{HiCutsを作るプログラム}){\FRecurs{rulelist,spfac,binth,range}}{
\KwData{ルールリスト,メモリ要件を調節する値,葉ノードの含むルールの上限数,フィールドの範囲}
\KwResult{構成した決定木}
\tcc{this is a comment to tell you that we will now really start code}
ルートノードを作る\;
\If(\tcc*[h]{a simple if but with a comment on the same line}){this is true}{
we do that, else nothing\;
\tcc{we will include other if so you can see this is possible}
\eIf{we agree that}{
we do that\;
}{
else we will do a more complicated if using else if\;
\uIf{this first condition is true}{
we do that\;
}
\uElseIf{this other condition is true}{
this is done\tcc*[r]{else if}
}
\Else{
in other case, we do this\tcc*[r]{else}
}
}
}
\tcc{now loops}
\For{\forcond}{
a for loop\;
}
\While{$i<n$}{
a while loop including a repeat--until loop\;
\Repeat{this end condition}{
do this things\;
}
}
They are many other possibilities and customization possible that you have to
discover by reading the documentation.
}
\end{algorithm}


\SetKwFunction{cut}{cut}%
\SetKwProg{CUT}{Function}{}{end}\SetKwFunction{FRecurs}{makeHicuts}
\begin{algorithm}
\CUT(\tcc*[h]{カットするプログラム}){\cut{rulelist,spfac,binth,range}}{
\KwData{ルールリスト,メモリ要件を調節する値,葉ノードの含むルールの上限数,フィールドの範囲}
\KwResult{カットしたルール}
\tcc{this is a comment to tell you that we will now really start code}
ルートノードを作る\;
\If(\tcc*[h]{a simple if but with a comment on the same line}){this is true}

}
\end{algorithm}

\end{document}
